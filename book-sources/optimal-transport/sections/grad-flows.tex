\section{Wasserstein (gradient) Flows}

The goal of this section is to expose the connection between optimal transport and certain evolutions over the space of probability distributions, particularly solutions to some PDEs and generative models using diffusion. The exposition is informal, focusing on intuition rather than rigorous proof. We work over the space $\mathcal{X} = \mathbb{R}^d$. It is also the opportunity to draw connexion with recent applications in ML, most notably analyzing the training dynamic of MLP, modeling deep transformers, and flow matching for generative models.

\subsection{Evolutions over the Space of Measures}

We consider the evolution $t \mapsto \alpha_t \in \mathcal{P}(\mathbb{R}^d)$. Such evolution can be described in a ``Lagrangian'' way as the advection of particles along a (time-dependent) vector field $v_t(x)$ in $\mathbb{R}^d$. At the particle level, this advection is governed by 
\begin{equation}
    \frac{\mathrm{d}x(t)}{\mathrm{d}t} = v_t(x(t)), \label{eq:lagrangian-advection}
\end{equation}
such that $x(0)$ is mapped to $x(t)$ by a ``transport'' mapping $T_t : x(0) \mapsto x(t)$. The fact that $\alpha_t$ is the density of advected particles implies $\alpha_t = (T_t)_\sharp \alpha_0$. For discrete measures, $\alpha_t = \frac{1}{n} \sum_{i=1}^n \delta_{x_i(t)}$, meaning each $x_i(t)$ solves \eqref{eq:lagrangian-advection}.

In the Eulerian interpretation, over the measure itself, the ODE for particles becomes the PDE
\begin{equation}
    \frac{\partial \alpha_t}{\partial t} + \mathrm{div}(v_t \alpha_t) = 0. \label{eq:eulerian-advection}
\end{equation}
Here, writing $\mathrm{div}(v_t \alpha_t)$ is an abuse of notation since divergence is strictly valid for densities. Instead, it refers to the measure defined by $\mathrm{div}\left(v_t \frac{\mathrm{d} \alpha_t}{\mathrm{d} x}\right) \mathrm{d}x$.

More rigorously, this PDE \eqref{eq:eulerian-advection} should be understood in the weak sense, allowing it to be defined even for discrete measures with particles evolving according to \eqref{eq:lagrangian-advection}. Specifically, for any smooth function $\varphi(x, t)$,
\begin{equation}
    \partial_t \int \varphi(x, t) \mathrm{d} \alpha_t(x) 
    + \int \langle v_t, \nabla_x \varphi(x, t) \rangle \mathrm{d} \alpha_t(x) = 0.
\end{equation}

It is important to note that for a given evolution $\alpha_t$, there are infinitely many possible choices of vector field $v_t$ satisfying
\begin{equation}
    \mathrm{div}(v_t \alpha_t) = -\partial_t \alpha_t. \label{eq:inverse-flow}
\end{equation}
This is because modifying $v_t$ by a divergence-free field does not affect the density evolution. Consequently, there are infinitely many particle evolutions that result in the same density. Reconstructing particle evolution from observed density evolution (sometimes called ``trajectory inference'') is thus an ill-posed inverse problem.

When employing optimal transport techniques, one typically assumes vector fields with minimal norms, selecting a specific $v_t$ via a least-effort principle. Using the Monge formulation for measures with density, we have
\begin{equation}
    \mathrm{W}_2(\alpha_t, \alpha_{t+\tau})^2 = \tau^2 \inf_{v_t} \left\{ \|v_t\|_{L^2(\alpha_t)}^2 = \int \|v_t\|^2 \mathrm{d} \alpha_t : (\mathrm{Id} + \tau v_t)_\sharp \alpha_t = \alpha_{t+\tau} \right\}, \label{eq:monge-field}
\end{equation}
where $\mathrm{W}_2$ is the Wasserstein-2 distance. By Brenier's theorem, the solution is the gradient of a function, $v_t = \nabla \psi_t$, and $|x|^2 + \tau \varphi(x)$ is convex. Thus, if $\tau$ is small and assuming $v_t = \nabla \psi_t$, \eqref{eq:inverse-flow} leads to the specific choice
\begin{equation}
    v_t = -\partial_t \Delta_{\alpha_t}^{-1}(\alpha_t),
\end{equation}
where $\Delta_\alpha(\varphi) := \mathrm{div}(\nabla \varphi \alpha)$ is a weighted Laplacian. 

In the following, we will use similar ideas to define $\alpha_t$, rather than assuming it is given.


\subsection{Wasserstein Gradient Flows}

We consider a function $f(\alpha)$ and seek a minimizing evolution $(\alpha_t)_t$. The general strategy of minimizing movement over a metric space is to construct a discrete-time evolution using an implicit Euler scheme:
\begin{equation}
    \alpha_{t+\tau} := \arg\min_\alpha \frac{1}{2 \tau} \mathrm{W}_2(\alpha_t, \alpha)^2 + f(\alpha). \label{eq:jko-discr}
\end{equation}

\paragraph{Euclidean gradient flows.}

If we restrict \eqref{eq:jko-discr} to finite dimensions and assume $\alpha_t = \delta_{x(t)}$ and $\alpha = \delta_x$ (single Dirac measures), this matches the implicit Euler scheme:
\begin{equation}
    x(t+\tau) := \arg\min_x \frac{1}{2 \tau} \|x - x(t)\|^2 + h(x),
\end{equation}
where $h(x) = f(\delta_x)$. Its solution is formally given by the implicit Euler formula:
\begin{equation}
    x(t+\tau) = (\mathrm{Id} + \tau \nabla h)^{-1}(x(t)).
\end{equation}
In contrast, the explicit Euler scheme is:
\begin{equation}
    x(t+\tau) = (\mathrm{Id} - \tau \nabla h)(x(t)) = x(t) - \tau \nabla h(x(t)).
\end{equation}
Both schemes converge as $\tau \to 0$ to:
\begin{equation}
    \dot{x}(t) = -\nabla h(x(t)). \label{eq:grad-flow-classical}
\end{equation}


\paragraph{Wasserstein gradient formula.}

The implicit Euler scheme has the advantage that it does not require $h$ or $f$ to be smooth. For $f$, this is crucial to handle evolution over arbitrary measures (with or without densities) seamlessly.

As $\tau \to 0$, under certain conditions on $f$, \eqref{eq:jko-discr} defines a continuous evolution $t \mapsto \alpha_t$. As discussed earlier, this evolution can be described as a Lagrangian evolution \eqref{eq:lagrangian-advection}. A key point is that it provides an explicit vector field $v_t$ (depending on $\alpha_t$), denoted as $\nabla_W f(\alpha)$, called the Wasserstein gradient. In the weak sense, $\alpha_t$ satisfies:
\begin{equation}
    \frac{\partial \alpha_t}{\partial t} + \mathrm{div}(-\nabla_W f(\alpha_t) \alpha_t) = 0. \label{eq:wassflow-pde}
\end{equation}
Here, $\nabla_W f(\alpha_t)(x) \in \mathbb{R}^d$ is a vector field and is a gradient (similar to the computation in \eqref{eq:monge-field}), computable as:
\begin{equation}
    \nabla_W f(\alpha) = \nabla_{\mathbb{R}^d} \varphi, \quad \text{where } \varphi := \delta f(\alpha).
\end{equation}
The function $\delta f(\alpha) \in \mathcal{C}(\mathbb{R}^d)$ is known as the first variation or Fréchet (directional) derivative, satisfying for any $\beta \in \mathcal{P}(\mathbb{R}^d)$:
\begin{equation}
    f((1-\tau)\alpha + \tau \beta) = f(\alpha + \tau \rho) = f(\alpha) + \tau \int [\delta f(\alpha)](x) \mathrm{d} \rho(x) + o(\tau),
\end{equation}
where $\rho = \beta - \alpha$ is a zero-mean measure.

%%%%
\paragraph{Heuristic derivation of the Wasserstein gradient formula.}

An heuristic way to see why~\eqref{eq:wassflow-pde} holds with this specific choice of vector field $\nabla_W f(\alpha)$ is first re-write \eqref{eq:jko-discr} as a minimization over displacement fields $v$ so that $\alpha = (\mathrm{Id} + \tau v)_\sharp \alpha_t$, i.e. considering
$$
	\min_{v} \frac{1}{2 \tau} \tau^2 \|v\|_{L^2(\alpha_t)}^2 + f( (\mathrm{Id} + \tau v)_\sharp \alpha_t ).
$$
Then we perform a first order Taylor expansion of this formulation using 
$$
	 (\mathrm{Id} + \tau v)_\sharp \alpha_t =  \alpha_t + \tau \mathrm{div}( v \alpha_t ) + o(\tau)
$$
$$
	f( (\mathrm{Id} + \tau v)_\sharp \alpha_t ) = f(\alpha_t) - \tau \int \delta f(\alpha_t) \mathrm{div}( v \alpha_t ) \mathrm{d} d x + o(\tau)
$$
$$
	 = f(\alpha_t) + \tau \int \langle \nabla_{\mathbb{R}^d} \delta f(\alpha_t)(x), v(x) \rangle \mathrm{d} \alpha_t(x) + o(\tau)
$$
to obtain the following first order expansion in $\tau$ of the problem minimized in  \eqref{eq:jko-discr}
$$
	 \min_{v} f(\alpha_t) +  \tau \int \Big[ \frac{1}{2} \|v(x)\|^2 + \langle \nabla_{W} f(\alpha_t)(x), v(x) \rangle \Big]\mathrm{d} \alpha_t(x) + o(\tau)
$$
We now detail examples of such Wasserstein gradient flows.

%%%
\paragraph{Discrete evolutions.}

If $f(\alpha)$ can be evaluated on discrete distributions and $\nabla_W$ is continuous in this case, the flow \eqref{eq:wassflow-pde} maintains the number of Dirac masses, $\alpha_t = \frac{1}{n} \sum_i \delta_{x_i(t)}$. The particles $X(t) := (x_i(t))_i$ evolve according to a system of coupled ODEs:
\begin{equation}
    \dot{X}(t) = -\nabla F(X), \label{eq:wassflows-particles}
\end{equation}
where $F(X) := f\left(\frac{1}{n} \sum_i \delta_{x_i}\right)$.
 
%%%
\paragraph{Linear Functionals.} The simplest example of flows is for linear functions
   \begin{equation}
       f(\alpha) = \int h(x) \mathrm{d} \alpha(x). \label{eq:linear-func}
   \end{equation}
   Here, $\delta f(\alpha) = h$ is a fixed function (independent of $\alpha$). The flow \eqref{eq:wassflow-pde} becomes:
   \begin{equation}
       \frac{\partial \alpha_t}{\partial t} + \mathrm{div}(-\nabla h \alpha_t) = 0.
   \end{equation}
   This implies particles move independently according to the usual gradient flow \eqref{eq:grad-flow-classical}.

%%%
\paragraph{Shannon Neg-Entropy.} A very different behavior is obtained by considering functions which require $\alpha_t$ to have a density, the canonical example being Shannon neg-entropy
   \begin{equation}
       f(\alpha) = \int \log\left(\frac{\mathrm{d} \alpha}{\mathrm{d} x}(x)\right) \mathrm{d} \alpha(x). \label{eq:entropy-func}
   \end{equation}
   Here, $\delta f(alpha) = \log\left(\frac{\mathrm{d} \alpha}{\mathrm{d} x}\right)$, so $\nabla_W f(\alpha) = \frac{\nabla \alpha}{\alpha}$ (often called the score). The flow \eqref{eq:wassflow-pde} becomes the heat equation:
   \begin{equation}
       \partial_t \alpha_t = \Delta(\alpha).
   \end{equation}
   Other entropy functionals lead to nonlinear diffusion equations. For example, power entropy:
   \begin{equation}
       f(\alpha) = \int g\left(\frac{\mathrm{d} \alpha}{\mathrm{d} x}\right) \mathrm{d} x,
   \end{equation}
   for a 1-D function $g(s)$, leads to nonlinear diffusions:
   \begin{equation}
       \frac{\partial \alpha_t}{\partial t} = \Delta(\tilde{g}(\alpha)),
   \end{equation}
   where $s g'(s) = \tilde{g}'(s)$. For example, $g(s) = s \log(s)$ corresponds to \eqref{eq:entropy-func}, while $g(s) = s^{p-1}/(p-1)$, $p > 1$, yields slow diffusion.

%%%
\paragraph{Interaction Energies.} In a similar spirit, to obtain non-linear evolutions, but without requiring the measure to have density, one can considering
   \begin{equation}
       f(\alpha) := \iint k(x, y) \mathrm{d} \alpha(x) \mathrm{d} \alpha(y). \label{eq:quadratic-func}
   \end{equation}
   For a symmetric kernel $k$:
   \begin{equation}
       \delta f(\alpha)(x) = 2 \int k(x, y) \mathrm{d} \alpha(y), \quad \nabla_W f(\alpha)(x) = 2 \int \nabla_x k(x, y) \mathrm{d} \alpha(y).
   \end{equation}
   For $\alpha_0 = \frac{1}{n} \sum_i \delta_{x_i}$, the flow \eqref{eq:wassflow-pde} implies particles $(x_i(t))_i$ obey:
   \begin{equation}
       \dot{x}_i(t) = -2 \sum_j \nabla k(x_i(t), x_j(t)).
   \end{equation}
In general, analyzing \eqref{eq:wassflow-pde} is challenging. A simple case is when $f$ is geodesically convex. Denoting $T$ as the optimal transport map from $\alpha_0$ to $\alpha_1$, the function $t \mapsto f(((1-t)\mathrm{Id} + tT)_\sharp \alpha_0)$ is convex. In this case, $\alpha_t$ is well-defined and converges to a global minimizer of $f$. This applies to linear \eqref{eq:linear-func}, quadratic \eqref{eq:quadratic-func} with convex $h(x)$ and $k(x, y)$, and Shannon entropy \eqref{eq:entropy-func}.

\subsection{Training Two-Layer MLPs as Wasserstein Flows}

In this section, we replace the variable $x$ with $\theta$ to align with customary notation in machine learning. 

We consider a two-layer MLP $g_\theta : u \in \mathbb{R}^d \to \mathbb{R}$ with $n$ neurons:
\begin{equation}
    g_\theta(u) := \frac{1}{n} \sum_i \psi(\theta_i, u), \quad \text{where } \psi(\theta_i, u) := a_i \langle u, w_i \rangle,
\end{equation}
and $\theta_i := (w_i, a_i) \in \mathbb{R}^{d+1}$ represents the parameters of a neuron. Importantly, these functions are invariant under permutations of the neurons. Thus, we can rewrite it using a probability distribution $\alpha = \frac{1}{n} \sum_i \delta_{\theta_i}$ as:
\begin{equation}
    G_\alpha(u) := \int_{\mathbb{R}^{d+1}} \psi(\theta, u) \, \mathrm{d} \alpha(\theta).
\end{equation}
This formulation has the advantage that $\alpha$ can represent a continuous density with an infinite number of neurons.

Now, we consider training the MLP by minimizing an empirical risk to predict labels $y_k \in \mathbb{R}$ from features $u_k \in \mathbb{R}^d$:
\begin{equation}
    \min_\alpha f(\alpha) := \frac{1}{N} \sum_{k=1}^N \ell(G_\alpha(u_k), y_k).
\end{equation}
Since $\alpha \mapsto G_\alpha$ is linear, if $\ell$ is convex, then $f$ is a convex function of $\alpha$. However, this observation is not particularly useful because $\alpha$ is infinite-dimensional, making standard minimization infeasible. The typical approach is to perform gradient descent on the neuron parameters, as described in \eqref{eq:wassflows-particles}:
\begin{equation}
    \dot{\theta} = -\nabla F(\theta), \quad \text{where } F(\theta) := f\left(\frac{1}{n} \sum_i \delta_{\theta_i}\right).
\end{equation}
This is equivalent to the PDE \eqref{eq:wassflow-pde} on the neuron density, where the Wasserstein gradient is used.

For $\ell(s, s') = \frac{1}{2}(s - s')^2$, the first variation of $f$ is:
\begin{equation}
    \delta f(\alpha)(\theta) = \frac{1}{N} \sum_{k=1}^N \left(\int \psi(\theta', u_k) \, \mathrm{d} \alpha(\theta') - y_k\right) \psi(\theta, u_k),
\end{equation}
which can be rewritten as:
\begin{equation}
    \delta f(\alpha)(\theta) = \int k(\theta, \theta') \, \mathrm{d} \alpha(\theta') + g(\theta),
\end{equation}
where the kernel and potential functions are:
\begin{align}
    k(\theta, \theta') &:= \frac{1}{N} \sum_k \psi(\theta, u_k) \psi(\theta', u_k), \\
    g(\theta) &:= -\frac{1}{N} \sum_k y_k \psi(\theta, u_k).
\end{align}
Thus, the Wasserstein gradient becomes:
\begin{equation}
    \nabla_W f(\alpha)(\theta) = \int \nabla_\theta k(\theta, \theta') \, \mathrm{d} \alpha(\theta') + \nabla_\theta g(\theta).
\end{equation}
This corresponds to a Wasserstein flow of the sum of a quadratic and a linear interaction potential. These functions are not geodesically convex because neither $k$ nor $-g$ are convex, making convergence analysis challenging.

A breakthrough was achieved by Chizat and Bach, who proved that if the initialization has enough Dirac masses, the flow cannot become stuck in local minima or saddle points. This result leverages the classical convexity of the function $f$ and the 1-homogeneity of $\psi((a, w), u)$ with respect to the external weights $a$.

\subsection{Evolution in Depth of Transformers}

We consider very deep transformers, focusing on a single-head attention mechanism for simplicity while ignoring MLP layers, layer normalization, causality, and masking. This framework is best suited to modeling encoders and vision transformers. 

After tokenization, embedding, and positional encoding, each input (from a set of tokens) is represented as a point cloud $(x_i)_{i=1}^n$ of $n$ points in the space of vectorized tokens. An attention layer with skip connection and rescaling by $1/T$ (where $T$ is the depth) defines a transformation of the tokens:
\begin{equation}
    x_i \mapsto x_i + \frac{1}{T} \sum_j \frac{e^{\langle Q x_i, K x_j \rangle} V x_j}{\sum_{\ell} e^{\langle Q x_i, K x_\ell \rangle}},
\end{equation}
where $\theta = (K, Q, V)$ are the parameters of the attention layer, represented by three matrices.

To handle an arbitrary number of tokens, we define $\alpha = \frac{1}{n} \sum_i \delta_{x_i}$ as the empirical measure of tokens and rewrite the transformer mapping as:
\begin{equation}
    x_i \mapsto x_i + \frac{1}{T} \Gamma_\theta[\alpha](x_i),
\end{equation}
where
\begin{equation}
    \Gamma_\theta[\alpha](x) := \int \frac{e^{\langle Q x, K y \rangle} V y \, \mathrm{d} \alpha(y)}{\int e^{\langle Q x, K z \rangle} \, \mathrm{d} \alpha(z)}.
\end{equation}
In terms of the evolution of the token distribution $\alpha$, this means $\alpha$ is pushed forward by the ``in-context'' mapping $\Gamma_{\theta_t}[\alpha]$, which depends on the context $\alpha$, the tokens, and the depth-dependent parameters $\theta_t$. Denoting $t \in [0, 1]$ as the depth and $\tau = 1/T$ as the step size, this gives:
\begin{equation}
    \alpha_{t+\tau} = (\mathrm{Id} + \tau \Gamma_{\theta_t}[\alpha_t])_\sharp \alpha_t.
\end{equation}
As $\tau \to 0$, this converges to the following conservation equation:
\begin{equation}
    \partial_t \alpha_t + \mathrm{div}(\alpha_t \Gamma[\alpha_t]) = 0.
\end{equation}
An interesting remark is that, when $V=KQ^T$, then $\Gamma[\alpha]$ is a gradient vector field, but it is not a gradient of a first variation, so that this PDE is not a Wasserstein gradient flow. 
%
This formulation was first introduced by Michael Sander in the Sinkformer's paper, modeling deep transformers as PDEs. The key challenge lies in understanding the training of the network, which corresponds to optimizing the parameters $(\theta_t)_t$. This remains an open problem.


\subsection{Generative Models via Flow Matching}

Generative models aim to build a transportation map $T$ between a reference distribution $\alpha$ (typically an isotropic Gaussian) and the target data distribution $\beta$. It is easy to see that such a map always exists for any $\beta$, but finding an explicit constructive method for $T$ is surprisingly non-trivial. 
%
Optimal transport is one approach to achieving this, but it is computationally expensive and raises questions about how to estimate it from samples. A recent idea, first introduced in diffusion models and later systematically developed in flow matching, is to obtain $T$ by integrating a time-dependent vector field $v_t$. This idea was introduced by Yaron Lipman and his collaborators. The key insight is that valid vector fields can be found via a simple conditional expectation, i.e., through a least squares problem, making the estimation feasible from finite samples of $\alpha$ and $\beta$.

We consider a coupling $\pi$ between $\alpha$ and $\beta$. The simplest choice is the ``independent'' coupling $\pi = \alpha \otimes \beta$. More complex couplings, such as an OT coupling, could also be considered, but they are computationally expensive. Using $\pi$, the interpolation between $\alpha = \alpha_0$ (at $t=0$) and $\beta = \alpha_1$ (at $t=1$) is obtained by push-forwarding using a linear interpolation:
\begin{equation}
    \alpha_t := (P_t)_\sharp \pi, \quad \text{where } P_t(x, y) := (1-t)x + ty. \label{eq:interp-coupling}
\end{equation}
If $\pi = \alpha \otimes \beta$ and $\alpha = \frac{1}{n} \sum_i \delta_{x_i}$, $\beta = \frac{1}{m} \sum_j \delta_{y_j}$, then $\alpha_t$ consists of $n \times m$ Dirac masses traveling in straight lines:
\begin{equation}
    \alpha_t = \frac{1}{nm} \sum_{i,j} \delta_{(1-t)x_i + ty_j}.
\end{equation}
If $\pi = (\mathrm{Id}, T)_\sharp \alpha$ is a Brenier-type coupling, then $\alpha_t = ((1-t)\mathrm{Id} + tT)_\sharp \alpha$ is the so-called McCann OT interpolation.

This interpolation is not directly useful for sampling from $\beta$, but it can be used to define a flow field $v_t$ so that the Eulerian advection equation holds:
\begin{equation}
    \frac{\partial \alpha_t}{\partial t} + \mathrm{div}(\alpha_t v_t) = 0. \label{eq:eulerian-advection}
\end{equation}
A valid $v_t$ can be found by solving the regression problem:
\begin{equation}
    \min_{(v_t)_t} \int \|v_t((1-t)x + ty) - (y - x)\|^2 \, \mathrm{d}\pi(x, y). \label{eq:flow-matching}
\end{equation}
Intuitively, this means that $v_t(x)$ at some point $x$ should be the average velocity of all trajectories passing through $x$
$$
		v_t(z) = \mathbb{E}_{(x, y) \sim \pi} \big[ y - x \, \big| \, z = (1-t)x + t y \big].
$$
Numerically, $v_t(x)$ can be parameterized by a neural network (e.g., a U-Net for vision tasks) and estimated using stochastic gradient descent on the objective in \eqref{eq:flow-matching}.

Once $v_t$ is estimated, integrating the ODE $\dot{x} = v_t(x)$ defines the transport map $T_t$, ensuring that $\alpha_t = (T_t)_\sharp \alpha$ produces the same interpolation as \eqref{eq:interp-coupling}, though with a different particle system. Instead of a coupling, this approach uses a deterministic map.
The sampling procedure consists in first drawing $X_0 \sim \alpha$, and then integrating the ODE $\dot{X}_t = v_t(X_t)$ starting with $X_{t=0} = X_0$. 
The resulting $X_{t=1}$ is distributed according to $\alpha_1 = \beta$.

It is customary to choose $\alpha$ as an isotropic Gaussian. In this case, $\alpha_t$ corresponds to a Gaussian convolution of $\beta$, and this method is equivalent to the diffusion model, up to an exponential reparameterization of time.
